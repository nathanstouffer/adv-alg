\documentclass{article}
\usepackage{../fasy-hw}
\usepackage{ wasysym }

%% UPDATE these variables:
\renewcommand{\hwnum}{1}
\title{Advanced Algorithms, Homework 1}
\author{Nathan Stouffer}
\collab{n/a}
\date{Due: 27 August 2020}

\begin{document}

\maketitle

%This homework assignment is due on 27 August 2020, and should be
%submitted as a single PDF file to D2L and to Gradescope.

%General homework expectations:
%\begin{itemize}
%    \item Homework should be typeset using LaTex.
%    \item Answers should be in complete sentences and proofread.
%\end{itemize}

\nextprob
\collab{n/a}

Answer the following questions:
\begin{enumerate}
    \item What is your elevator pitch?  Describe yourself in 1-2
                sentences.
    \item What was your favorite CS class so far, and why?
    \item What was your least favorite CS class so far, and why?
    \item Why are you interested in taking this course?
    \item What is your biggest academic or research goal for this semester (can
        be related to this course or not)?
    \item What do you want to do after you graduate?
    \item What was the most challenging aspect of your coursework last semester
        after the university transitioned to online?
    \item What went well last semester for you after the university transitioned
        to online?
\end{enumerate}

\paragraph{Answer}

% ============================================

\begin{enumerate}
    \item I am a senior in math and computer science.
    I really enjoy thinking, running, and skiing.
    \item So far, my favorite computer science class has been computer graphics.
    I really liked how much math and programming there was.
    The math appealed to the side of me that enjoys thinking about problems and the programming appealed to the side of me that enjoys building solutions.
    \item So far, my least favorite computer science class was computer architecture.
    I did not find the content very exciting.
    I prefer using the computer at higher level than messing around with resistors.
    \item I am interested in taking advanced algorithms because I think it is a cool area.
    I like coming up with clean, fast solutions to problems and I think this will be an opportunity to do so.
    Another reason that I am interested in taking this course is because there are a lot of cool applications!
    \item This is more of a year long goal, but I would like to publish a paper this year.
    \item I am unsure of what I would like to do after I graduate.
    I am considering graduate school but I am also thinking about working in industry.
    \item After school went online, the most challenging portion of my coursework was completing quality work in every one of my courses.
    It was sometimes difficult to stay motivated in courses that I was less interested in.
    \item After going online, I was very good at staying organized and on top of my schoolwork.
\end{enumerate}

% ============================================

\nextprob
\collab{n/a}

Please do the following:
\begin{enumerate}
    \item Write this homework in LaTex.
        Note: if you have not used LaTex before and this is an
        issue for you, please contact the instructor or TA.
    \item Update your photo on D2L to be a recognizable headshot of you.
    \item Sign up for the class discussion board.
\end{enumerate}

\paragraph{Answer}

% ============================================

I have completed the above tasks.

% ============================================


\nextprob
\collab{none}

    In this class,
    please properly cite all resources that you use.
    To refresh your memory on what plagiarism is,
    please
    complete the plagiarism tutorial found here:
    \url{http://www.lib.usm.edu/plagiarism_tutorial}.
    If you have observed plagiarism or cheating in a classroom (either as an
    instructor or as a student), explain the situation and how it made you
    feel.  If you have not experienced plagiarism or cheating or if you would
    prefer not to reflect on a personal experience, find a news
    article about plagiarism or cheating and explain how you would feel if you
    were one of the people involved.

\paragraph{Answer}

% ============================================

I saw cheating while in my high school math class.
Some students copied their homework answers from other students.
I felt rather indifferent about their cheating.
I remember thinking that they were only cheating themselves and that it would catch up with them eventually.

% ============================================



\nextprob
\collab{none}

Prove the following statement: Every tree with one or more nodes/vertices has
exactly $n-1$ edges.

\paragraph{Answer}

% ============================================

We must show that every tree with one or more nodes has exactly $n-1$ edges.
We take a tree to be a connected, undirected graph with no cycles.
We will show this by using induction. \parspace
Base Case: Let's begin with the base case of a tree with a single node.
Such a tree has no edges.
Does our claim hold?
Certainly: $n-1 = 1-1 = 0$.
So the base case is proved. \parspace
Inductive assumption: We now assume that there exists some $k \in \N$ such that every tree with $k$ nodes has $k-1$ edges. \parspace
Inductive step: Now can we show that every tree with $k+1$ nodes has $k$ edges?
Let's begin with a tree $T_k$ with $k$ nodes.
From our inductive assumption, we know that $T_k$ has $k-1$ edges.
We then add one node to $T_k$ to form $T_{k+1}$. \parspace
When adding node $k+1$, we are constrained to add exactly one edge.
We must add a non-zero number of edges because $T_{k+1}$ must be connected.
But we cannot add more than one ege since $T_k$ is connected and adding more than one edge would form a cycle, meaning $T_{k+1}$ would not be a tree.
So we can only add one edge when constructing $T_{k+1}$.
So $T_{k+1}$ must have $k-1+1=k$ edges, which is what we needed to show in the inductive step. \parspace
Thus, by induction, we have shown that every tree with one or more nodes must have exactly $n-1$ edges.

% ============================================



\nextprob
\collab{none}

Use the definition of big-O notation to prove that $f(n)=n^2 + 3n +2$ is
$O(n^2)$.

\paragraph{Answer}

% ============================================

To show that $f(n) = n^2 + 3n + 2 = O(n^2)$, we must show the existence of $n_0 \in \N$ and $c \in \R$ with $ c > 0$ such that $f(n) \leq c n^2$  for all $n \geq n_0$.
Towards finding such constants, we observe that
$$ f(n) = n^2 + 3n + 2 \leq n^2 + 3n^2 + 2n^2 = 6n^2 \hspace{1em} \forall n \in \N $$
We choose $n_0 = 1$ and $c = 6$.
Thus, $f(n) = n^2 + 3n + 2 $ is $O(n^2)$.

% ============================================



\nextprob
\collab{none}

Consider the \textsc{RightAngle} algorithm on page 8 of the textbook.
\begin{enumerate}
    \item When we design an algorithm, we design the algorithm to solve a
        problem or answer a question.  What is the problem that this algorithm
        solves?
    \item Prove that the algorithm terminates.
\end{enumerate}

\paragraph{Answer}

% ============================================

\begin{enumerate}
    \item Given a line $l$ and a point $p \in l$, the \textsc{RightAngle} algorithm draws a line that passes through $p$ and is perpendicular to $l$.
    \item We must now show that \textsc{RightAngle} terminates.
    I found that \textsc{RightAngle} terminates in all but one case. \parspace
    Case $A \neq P$: In this case, \textsc{RightAngle} terminates if each of its steps terminates.
    Steps one certainly terminates since it just a selection of a point on a line.
    Step two terminates because $P \neq A$ so \textsc{Circle(P,A)} can be drawn and there are two intersection points with $l$.
    Step three terminates since both circles can be drawn and they have two intersection points (because they are circles of radius $r$ with distance between centers of $r$).
    Step four terminates because drawing a line between two distinct points is always possible. \parspace
    Case $A = P$: This case cannot terminate because \textsc{Circle(P,P)} cannot be drawn.
\end{enumerate}

% ============================================



\nextprob
\collab{none}

Consider the following statement: If $a$ and $b$ are both even numbers, then $ab$ is
an even number.
\begin{enumerate}
    \item What is the definition of an odd number?
    \item What is the definition of an even number?
    \item What is the contrapositive of this statement?
    \item What is the converse of this statement?
    \item Prove this statement.
\end{enumerate}

\paragraph{Answer}

% ============================================

We now give answers to the above questions.

\begin{enumerate}
    \item The following is the definition of an odd number.
    If an integer $n$ can be written as $ n = 2*k + 1 $ for some integer $k$, then $n$ is said to be odd.
    \item The following is the definition of an even number.
    If an integer $n$ can be written as $ n = 2*k$ for some integer $k$, then $n$ is said to be even.
    \item The contrapositive of ``If $a$ and $b$ are both even numbers, then $ab$ is an even number'' is ``If $ab$ is an odd number, then either $a$ or $b$ must be odd.''
    \item The converse of ``If $a$ and $b$ are both even numbers, then $ab$ is an even number'' is ``If $ab$ is an even number, then $a$ and $b$ are both even numbers.''
    \item We now prove the statement ``If $a$ and $b$ are both even numbers, then $ab$ is an even number.''
    We prove this directly.

    Since $a$ and $b$ are both even, there exist integers $k,j$ such that $a = 2k$ and $b = 2j$.
    By substitution, $ab = (2k) (2j) = 2(2kj) $.
    Let $n = 2kj$, we know that $n \in \Z$ by closure of integers with multiplication.
    We also know that $2n$ is even and that $2n = ab$.
    Since $2n$ is even, $ab$ must be even as well.
    So, we have shown that given two even numbers $a$ and $b$, their product $ab$ must also be even.
\end{enumerate}


% ============================================



\end{document}
