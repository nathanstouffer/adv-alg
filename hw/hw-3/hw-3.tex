\documentclass{article}
\usepackage{../fasy-hw}
\usepackage{ wasysym }

%% UPDATE these variables:
\renewcommand{\hwnum}{3}
\title{Advanced Algorithms, Homework \hwnum}
\author{Nathan Stouffer \and Kevin Browder}
\collab{n/a}
\date{due: 17 September 2020}

\begin{document}

\maketitle

This homework assignment is due on 3 September 2020, and should be
submitted as a single PDF file to to Gradescope.

General homework expectations:
\begin{itemize}
    \item Homework should be typeset using LaTex.
    \item Answers should be in complete sentences and proofread.
    \item This homework can be submitted as a group.
\end{itemize}

\nextprob
\collab{Nathan Stouffer and Kevin Browder}

Work in a group of size $\geq 2$.  Explain your strategy for working in a group.

\paragraph{Answer}

% ============================================

TODO: your answer goes between these lines

% ============================================

\nextprob
\collab{Nathan Stouffer and Kevin Browder}

Your group should make at least five contributions to the Piazza board.  A
contribution can be either asking a relevant question, responding to another
student's question, responding to an instructor's question, or choosing a
question from Chapter 1 and attempting to solve it, then  describing where you
get stuck in answering it.

\paragraph{Answer}

% ============================================

Our groups contributions are:
\begin{enumerate}
    \item (TODO: state the problem number, name of poster, and date/time). TODO:
        copy the post here.
    \item (TODO: state the problem number, name of poster, and date/time). TODO:
        copy the post here.
    \item (TODO: state the problem number, name of poster, and date/time). TODO:
        copy the post here.
    \item (TODO: state the problem number, name of poster, and date/time). TODO:
        copy the post here.
    \item (TODO: state the problem number, name of poster, and date/time). TODO:
        copy the post here.
\end{enumerate}

% ============================================

\nextprob
\collab{Nathan Stouffer and Kevin Browder}

Give the algorithm for binary search, using a for loop and no recursion.

\begin{enumerate}
    \item Describe the problem in your own words, including
        describing what the input and output is.
    \item Describe, in paragraph form, the algorithm you propose.
    \item Provide this algorithm in the algorithm environment.
    \item Use a decrementing function to prove that the loop terminates.
    \item What is the loop invariant? Provide the proof.
\end{enumerate}

\paragraph{Answer}

% ============================================

TODO: your answer goes between these lines

% ============================================

\nextprob
\collab{Nathan Stouffer and Kevin Browder}

Chapter 2, Problem 1b (Generalized \textsc{SubsetSum}).
\begin{enumerate}
    \item Describe the problem in your own words, including
        describing what the input and output is.
    \item Describe, in paragraph form, the algorithm you propose.
    \item Provide this algorithm in the algorithm environment.
    \item What is the runtime of your algorithm?
    \item Prove partial correctness (that if your algorithm terminates, it is
        correct).
\end{enumerate}



\paragraph{Answer}

% ============================================

TODO: your answer goes between these lines

% ============================================



\nextprob
\collab{Nathan Stouffer and Kevin Browder}

Describe two different data structures that you can use to store a graph.
Please give a complete description (i.e., a response of ``an array'' will not
suffice).

\paragraph{Answer}

% ============================================

TODO: your answer goes between these lines

% ============================================


\nextprob
\collab{Nathan Stouffer and Kevin Browder}

Walk through the exponential time Longest Increasing Subsequence (LIS) algorithm
on page 108 for the input: $\left[ 1, 7, 6, 11, 3, 11 \right]$.

Walk through the algorithm using the Dynamic Programming algorithm present in
Section 3.6.

\paragraph{Answer}

% ============================================

TODO: your answer goes between these lines

% ============================================

\nextprob
\collab{Nathan Stouffer and Kevin Browder}

What is the closed form of the following recurrence relations?  Use Master's
theorem to justify your answers:
\begin{enumerate}
    \item $T(n) = 16 T(n/4) + \Theta(n)$
    \item $T(n) = 2 T(n/2) + n \log{n}$
    \item $T(n) = 6 T(n/3) + n^2 \log{n}$
    \item $T(n) = 4 T(n/2) + n^2$
    \item $T(n) = 9 T(n/3) + n$
\end{enumerate}
Note: we assume that $T(1)=\Theta(1)$ whenever it is not explicitly given.

\paragraph{Answer}

% ============================================

TODO: your answer goes between these lines

% ============================================

\nextprob
\collab{Nathan Stouffer and Kevin Browder}

\emph{The skyline problem:} You are in Camden, NJ waiting for the ferry across the river to
get into Philadelphia, and are looking at the skyline.  You take a photo, and notice that each building
has the silhouette of a rectangle.  Suppose you  represent each building $b$ as a
triple $(x_b^{(1)},x_b^{(2)},y_b)$, where the building can be seen from $x_b^{(1)}$ to $x_b^{(2)}$
horizontally and has a height of $y_b$.  Let $\mathtt{rect(b)}$ be the set of
points inside this rectangle (including the boundary).  Let $\mathtt{buildings}$
be a set of $n$ such triples representing buildings. Design an algorithm that takes $\mathtt{buildings}$ as input, and
returns the skyline, where the skyline is a sequence of~$(x,y)$ coordinates
defining $\cup_{b \in \mathtt{buildings}} \mathtt{rect}(b)$.  The output should
start with $(\min_b{x_b^{(1)}},0)$ and end with $(\max_b{x_b^{(1)}},0)$.

\begin{enumerate}
    \item Describe the problem in your own words, including
        describing what the input and output is.
    \item Describe, in paragraph form, the algorithm you propose.
    \item Provide this algorithm in the algorithm environment.
    \item What is the runtime of your algorithm? If you do not know, either give
        the tightest bounds you know, or provide a decrementing function to show
        that it does terminate.
    \item Prove partial correctness (that if your algorithm terminates, it is
        correct).
\end{enumerate}



\paragraph{Answer}

% ============================================

TODO: your answer goes between these lines

% ============================================



\end{document}
