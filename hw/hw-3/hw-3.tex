separately\documentclass{article}
\usepackage{../fasy-hw}
\usepackage{ wasysym }

%% UPDATE these variables:
\renewcommand{\hwnum}{3}
\title{Advanced Algorithms, Homework \hwnum}
\author{Nathan Stouffer \and Kevin Browder}
\collab{n/a}
\date{due: 17 September 2020}

\begin{document}

\maketitle

This homework assignment is due on 3 September 2020, and should be
submitted as a single PDF file to to Gradescope.

General homework expectations:
\begin{itemize}
    \item Homework should be typeset using LaTex.
    \item Answers should be in complete sentences and proofread.
    \item This homework can be submitted as a group.
\end{itemize}

\nextprob
\collab{Nathan Stouffer and Kevin Browder}

Work in a group of size $\geq 2$.  Explain your strategy for working in a group.

\paragraph{Answer}

% ============================================

Our strategy for working in a group: we meet early in the week and work over the problems on scratch paper very informally. Doing this early will give us some extra time if there are any very difficult/time consuming problems Once we are satisfied with our work we do the LaTeX separately on our own time and use Git to collaborate on it. We then meet again before the assignment is due and go over the finished document and make sure everything looks good and we didn't make any mistakes when doing the LaTeX. 

% ============================================

\nextprob
\collab{Nathan Stouffer and Kevin Browder}

Your group should make at least five contributions to the Piazza board.  A
contribution can be either asking a relevant question, responding to another
student's question, responding to an instructor's question, or choosing a
question from Chapter 1 and attempting to solve it, then  describing where you
get stuck in answering it.

\paragraph{Answer}

% ============================================

Our groups contributions are:
\begin{enumerate}
    \item (3-4, Kevin Browder, and 7:13-9/14/2020). As we learned in CS Theory a subset sum problem, which doesn't have a weight array is NP-Complete. Does the weight component of this problem change that? We think that it does because computing a maximum seems like it requires computing all the options to know for sure that we have the maximum.
    \item (TODO: state the problem number, name of poster, and date/time). TODO:
        copy the post here.
    \item (TODO: state the problem number, name of poster, and date/time). TODO:
        copy the post here.
    \item (TODO: state the problem number, name of poster, and date/time). TODO:
        copy the post here.
    \item (TODO: state the problem number, name of poster, and date/time). TODO:
        copy the post here.
\end{enumerate}

Problem number 3-3 (hw 3), Poster: Nathan Stouffer, Date: 9/15/2020 9:34 am
Hello everyone,

Some languages (like java, c, c++ ...) allow any boolean condition to be a test for whether the loop should continue running and also allow any incrementing function. However, the examples that I have seen in this class seem to have a for loop running over a set of integers. Does the real-ram model allow for for loops that have a condition like end > beg with some specified incrementing function?

The reason I ask is because of hw problem 3-3. The question asks that we write an algorithm for binary search using a for loop instead of recursion. If we must run over a set of integers, we can compute an upper bound on how many iterations the for loop will run so it is possible to create a correct for loop. I also think proving termination would be easier when running over a set of integers. However, the first option would be able to exit the for-loop before reaching the upper bound of iterations. This would not improve worst case run time but it would affect the average case run time. Let me know what we are allowed to do with this model of computation.

Another idea would be to break out the loop if a certain condition is true. Is this allowable with our model of computation?
% ============================================

\nextprob
\collab{Nathan Stouffer and Kevin Browder}

Give the algorithm for binary search, using a for loop and no recursion.

\begin{enumerate}
    \item Describe the problem in your own words, including describing what the input and output is.
    \item Describe, in paragraph form, the algorithm you propose.
    \item Provide this algorithm in the algorithm environment.
    \item Use a decrementing function to prove that the loop terminates.
    \item What is the loop invariant? Provide the proof.
\end{enumerate}

\paragraph{Answer}

% ============================================

\begin{enumerate}
    \item The problem that binary search solves is returning the location of a target value in an array.
    The input to the problem must be a sorted array of comparable items paired with a target value of the same type.
    The output will either by the index of the target value or -1 to flag that the target is not in the array.
    \item
    \item Here is the algorithm.
    \begin{algorithm}
    	\textsc{BinarySearch}(A[1..n], targ) \\
    	1.  \hspace{0em}   beg $\leftarrow$ 1 \\
    	2.  \hspace{0em}   end $\leftarrow n$ \\
        3.  \hspace{0em}   indx $\leftarrow$ -1 // assume value is not in array \\
    	4.  \hspace{0em}   for $1.. \lceil log_2(len(A)) \rceil$ \\
    	5.  \hspace{2em}       mid $\leftarrow \lfloor (\text{beg}+\text{end})/2 \rfloor$ \\
    	6.  \hspace{2em}	   if (A[mid] = targ) \\
    	7.  \hspace{4em}	       indx $\leftarrow$ mid \\
    	8.  \hspace{2em}	   if (A[mid] $>$ targ) \\
    	9.  \hspace{4em}		   end $\leftarrow$ mid - 1 \\
    	10. \hspace{2em}	   if (A[mid] $<$ targ) \\
        11. \hspace{4em}        beg $\leftarrow$ mid + 1 \\
        12. \hspace{0em} return indx
    \end{algorithm}
    \item prove termination
    \item Our loop invariant is that $arr[beg] < targ < arr[end]$.

\end{enumerate}

% ============================================

\nextprob
\collab{Nathan Stouffer and Kevin Browder}

Chapter 2, Problem 1b (Generalized \textsc{SubsetSum}).
\begin{enumerate}
    \item Describe the problem in your own words, including
        describing what the input and output is.
    \item Describe, in paragraph form, the algorithm you propose.
    \item Provide this algorithm in the algorithm environment.
    \item What is the runtime of your algorithm?
    \item Prove partial correctness (that if your algorithm terminates, it is
        correct).
\end{enumerate}


\paragraph{Answer}

% ============================================

\begin{enumerate}
    \item For the Generalized \textsc{SubsetSum} problem, our input has two parts.
    First, we have two equally sized arrays $X$ and $W$ containing positive integers paired with another positive integer $T$.
    Each value of $X$ has a corresponding weight in $W$ that can be found at the same index (ie weight for $X[i]$ can be found at $W[i]$).
    The second part of the input is an array called $I$ (which has length $len(X)$ and begins consisting entirely of 0s) paired with an index $i$ (which starts at 1).
    If it exists, our task is to find the subset of $X$ that sums to $T$ with the heaviest weight.
    If no subset of $X$ sums to $T$, we will return $-\infty$.
    Our output will either be $-\infty$ or a positve integer (the weight of the heaviest subset that sums to $T$).
    \item algorithm description \\
    The array $I$ contains activations for whether to include the $i^{th}$ element of $X$ in the subset (0 means don't include, 1 means include).
    The index i denotes the first index of $I$ that has not been yet been called.
    \item Here is the algorithm.
    \begin{algorithm}
        \textsc{SubsetSum}$(X, W, T, I, i)$ \\
        1.  \hspace{0em} if (\textsc{Sum}$(X, I) > T$) \\
        2.  \hspace{2em}     return $-\infty$ \\
        3.  \hspace{0em} if (\textsc{Sum}$(X, I) = T$) \\
        4.  \hspace{2em}     return \textsc{Weight}($W, I$) \\
        5.  \hspace{0em} skip $\leftarrow$ \textsc{SubsetSum}($X, W, T, I, i+1$) \\
        6.  \hspace{0em} $I[i] \leftarrow 1$ \\
        7.  \hspace{0em} take $\leftarrow$ \textsc{SubsetSum}($X, W, T, I, i+1$) \\
        8.  \hspace{0em} return max(skip, take) \\

        \textsc{Sum}(X, I) \\
        1.  \hspace{0em} sum $\leftarrow$ 0 \\
        2.  \hspace{0em} for i in 1..len(X) \\
        3.  \hspace{2em}     sum $\leftarrow$ sum + $X[i]*include[i]$ \\
        4.  \hspace{0em} return sum \\

        \textsc{Weight}(W, I) \\
        1.  \hspace{0em} weight $\leftarrow$ 0 \\
        2.  \hspace{0em} for i in 1..len(W) \\
        3.  \hspace{2em}     weight $\leftarrow$ weight + $W[i]*include[i]$ \\
        4.  \hspace{0em} return weight
    \end{algorithm}
    \newpage
    \item Towards giving the runtime of our algorithm, we start by giving the run times of \textsc{Sum} and \textsc{Weight}.
    Let $n$ be the length of $X$ (which matches the lengths of $W$ and $I$).
    Both \textsc{Sum} and \textsc{Weight} run in $\Theta(n)$ (since they have constant time operations that run $n$ times).
    Now we give the worst case recurrence relation for \textsc{SubsetSeum}: $T(n) = \Theta(n) + \Theta(n) + T(n-1) + O(1) + T(n-1) = 2*T(n-1) + \Theta(n)$.
\end{enumerate}

% ============================================



\nextprob
\collab{Nathan Stouffer and Kevin Browder}

Describe two different data structures that you can use to store a graph.
Please give a complete description (i.e., a response of ``an array'' will not
suffice).

\paragraph{Answer}

% ============================================

TODO: your answer goes between these lines

% ============================================


\nextprob
\collab{Nathan Stouffer and Kevin Browder}

Walk through the exponential time Longest Increasing Subsequence (LIS) algorithm
on page 108 for the input: $\left[ 1, 7, 6, 11, 3, 11 \right]$.

Walk through the algorithm using the Dynamic Programming algorithm present in
Section 3.6.

\paragraph{Answer}

% ============================================

TODO: your answer goes between these lines

% ============================================

\nextprob
\collab{Nathan Stouffer and Kevin Browder}

What is the closed form of the following recurrence relations?  Use Master's
theorem to justify your answers:
\begin{enumerate}
    \item $T(n) = 16 T(n/4) + \Theta(n)$
    \item $T(n) = 2 T(n/2) + n \log{n}$
    \item $T(n) = 6 T(n/3) + n^2 \log{n}$
    \item $T(n) = 4 T(n/2) + n^2$
    \item $T(n) = 9 T(n/3) + n$
\end{enumerate}
Note: we assume that $T(1)=\Theta(1)$ whenever it is not explicitly given.

\paragraph{Answer}

% ============================================

TODO: your answer goes between these lines

% ============================================

\nextprob
\collab{Nathan Stouffer and Kevin Browder}

\emph{The skyline problem:} You are in Camden, NJ waiting for the ferry across the river to
get into Philadelphia, and are looking at the skyline.  You take a photo, and notice that each building
has the silhouette of a rectangle.  Suppose you  represent each building $b$ as a
triple $(x_b^{(1)},x_b^{(2)},y_b)$, where the building can be seen from $x_b^{(1)}$ to $x_b^{(2)}$
horizontally and has a height of $y_b$.  Let $\mathtt{rect(b)}$ be the set of
points inside this rectangle (including the boundary).  Let $\mathtt{buildings}$
be a set of $n$ such triples representing buildings. Design an algorithm that takes $\mathtt{buildings}$ as input, and
returns the skyline, where the skyline is a sequence of~$(x,y)$ coordinates
defining $\cup_{b \in \mathtt{buildings}} \mathtt{rect}(b)$.  The output should
start with $(\min_b{x_b^{(1)}},0)$ and end with $(\max_b{x_b^{(1)}},0)$.

\begin{enumerate}
    \item Describe the problem in your own words, including
        describing what the input and output is.
    \item Describe, in paragraph form, the algorithm you propose.
    \item Provide this algorithm in the algorithm environment.
    \item What is the runtime of your algorithm? If you do not know, either give
        the tightest bounds you know, or provide a decrementing function to show
        that it does terminate.
    \item Prove partial correctness (that if your algorithm terminates, it is
        correct).
\end{enumerate}



\paragraph{Answer}

% ============================================

TODO: your answer goes between these lines

% ============================================



\end{document}
