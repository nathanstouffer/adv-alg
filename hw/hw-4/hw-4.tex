\documentclass{article}
\usepackage{../fasy-hw}
\usepackage{ wasysym }

%% UPDATE these variables:
\renewcommand{\hwnum}{4}
\title{Advanced Algorithms, Homework \hwnum}
\author{Nathan Stouffer}
\collab{n/a}
\date{due: 6 October 2020}

\begin{document}

\maketitle

This homework assignment should be
submitted as a single PDF file to to Gradescope.

General homework expectations:
\begin{itemize}
    \item Homework should be typeset using \LaTeX.
    \item Answers should be in complete sentences and proofread.
    \item This homework can be submitted as a group.
\end{itemize}

\nextprob
\collab{TODO}

You should make at least ten contributions to the Piazza board
discussing the solutions to Problems in Chapter 3 of the textbook.  Your
contribution does not have to be a complete solution.  It can be any element of
a full solution to a problem requiring an algorithm as an answer.  (For this
question, the outcomes are: insufficient posts (-1), low pass (+1), pass (+3),
and high pass (+5).

As a reminder, a full solution to a textbook problem will have the following elements:
\begin{enumerate}
    \item Describe the problem in your own words, including
        describing what the input and output is.
    \item Describe, in paragraph form, the algorithm you propose.
    \item Provide a nicely formatted algorithm to solve the problem.
    \item Use a decrementing function to prove that algorithm terminates.
    \item Give the runtime with justification.
    \item If there is a loop or recursion, what is the loop/recursion invariant? Provide the proof.
\end{enumerate}

\paragraph{Answer}

% ============================================

Our groups contributions are:
\begin{enumerate}
    \item (TODO: state the problem number, and date/time). TODO:
        copy the post here.
    \item ...
\end{enumerate}

% ============================================

\nextprob
\collab{TODO}

Choose one of the Chapter 3 problems discussed in Piazza and provide a solution
in your own words.  This should be a polished solution.

\paragraph{Answer}
% ============================================

TODO: your answer goes between these lines

% ============================================

\nextprob
\collab{TODO}

For this problem, choose either the Edit distance algorithm (Section 3.7) or the
Subset Sum problem (Section 3.8). Look at three different sources (including the
textbook) that describe
and analyze the same algorithm. In one to two pages, describe the similarities
and differences in the presentation and analysis of the algorithms.

\paragraph{Answer}

% ============================================

We choose to look at the Edit Distance Algorithm from Section 3.7 of the textbook. Our three sources were the textbook, GeeksforGeeks\footnote{Edit Distance: DP-5, GeeksforGeeks, January 13, 2020, https://www.geeksforgeeks.org/edit-distance-dp-5/.} and a Stanford lecture\footnote{Dan Jurafsky, “Minimum Edit Distance,” Stanford University, n.d., https://web.stanford.edu/class/cs124/lec/med.pdf.}. All  three presentations begin with a quick definition of the what the algorithm actually does in words. The textbook goes into much more detail than the other two sources but all sources include a quick example showing the algorithm working and the resulting output. Stanford's presentation also touches on practical implementation of Edit distance including spell check, computational biology and others. After this introduction GeeksforGeeks and the textbook dive into the subproblems and recursive approach to this problem. The Stanford article touches on the Nieve approach briefly but not in the depth that the other two look at it. GeeksforGeeks gives fully functional

% ============================================

\end{document}
