\documentclass{article}
\usepackage{../fasy-hw}
\usepackage{ wasysym }

%% UPDATE these variables:
\renewcommand{\hwnum}{5}
\title{Advanced Algorithms, Homework \hwnum}
\author{Nathan Stouffer}
\collab{n/a}
\date{due: 26 October 2020}

\begin{document}

\maketitle

In any question that you are expected to provide an algorithm, you are
expected to provide:
\begin{enumerate}
    \item Describe the problem in your own words, including
        describing what the input and output is.
    \item Describe, in paragraph form, the algorithm you propose.
    \item Provide a nicely formatted algorithm to solve the problem.
    \item Use a decrementing function to prove that algorithm terminates.
            OR  Give the runtime with justification.
    \item Prove partial correctness.  In other words, if there is a loop or
        recursion, what is the loop/recursion invariant? Provide the proof.
        (Note: you only need to do this for the outer-most loop if there are
        nested loops).
\end{enumerate}



\nextprob
\collab{TODO}

Walk through Kruskal's algorithm, using the graph in Figure 7.7 (left) of the
textbook.  Label the center vertex $a$, the other red vertex $b$, and the
remainder $c$ through $g$ in counter-clockwise order.  You should use the
union-find data structure, with both ``heuristics.''

\paragraph{Answer}

% ============================================

TODO: your answer goes between these lines

% ============================================

\nextprob
\collab{TODO}

Chapter 3, Question 9 (Palindromes)

A palindrome is any string that is exactly the same as its reversal.
\begin{enumerate}[label=(\alph*)]
    \item Describe and analyze an algorithm to find the length of the longest subsequence of a given string that is also a palindrome.
    \item Describe and analyze an algorithm to find the length of the shortest supersequence of a given string that is also a palindrome.
    \item Any string can be cecomposed into a sequence of palindromes.
    Describe and analyze an efficient algorithm to find the smallest number of palindromes that make up a given input string.
    \item Describe and analyze an efficient algorithm to find the largest integer $k$ such that a given string can be split into palindromes of length at least $k$.
    \item Describe and analyze and efficient algorithm to find the number of different ways that a given string can be decomposed into palindromes.
    \item A metapalindrome si a decomposition of a string into a sequence of palindromes, such that the sequence of palindrome lengths is itself a palindrome.
    Describe and analyze an efficient algorithm to find the length of the shortest metapalindrome for a given string.
\end{enumerate}

\paragraph{Answer}

% ============================================

TODO: your answer goes between these lines

% ============================================

\nextprob
\collab{TODO}

Chapter 7, Question 1 (Shortest and Longest Edges in Cycle)

Let $G = (V, E)$ be an arbitrary connected graph with weighted edge.
\begin{enumerate}[label=(\alph*)]
    \item Prove that for any cycle in $G$, the minimum spanning tree of $G$ excludes the maximum-weight edge in that cycle.
    \item Prove or disprove: The minimum spanning tree of $G$ includes the minimum-weight edge in every cycle in $G$.
\end{enumerate}

\paragraph{Answer}

% ============================================

TODO: your answer goes between these lines

% ============================================


\nextprob
\collab{TODO}

Chapter 8, Question 3 (Weights on Vertices)

Suppose we are given an undirected graph $G$ in which every vertex has a positive weight.
\begin{enumerate}[label=(\alph*)]
    \item Describe and analyze an algorithm to find a spanning tree of $G$ with minimum total weight.
    \item Describe and analyze an algorithm to find a path in $G$ from one given vertex $s$ to another given vertex $t$ with minimum total weight.
\end{enumerate}

\paragraph{Answer}

% ============================================

TODO: your answer goes between these lines

% ============================================




\nextprob
\collab{TODO}

Describe a ``real-life'' problem that can be modeled as:

\begin{enumerate}
    \item An undirected graph.
    \item A directed, weighted graph.
    \item A tree.
    \item A forest.
\end{enumerate}

\paragraph{Answer}

% ============================================

TODO: your answer goes between these lines
% ============================================


\end{document}
