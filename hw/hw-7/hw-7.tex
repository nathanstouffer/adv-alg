\documentclass{article}
\usepackage{../fasy-hw}
\usepackage{ wasysym }

%% UPDATE these variables:
\renewcommand{\hwnum}{7}
\title{Advanced Algorithms, Homework \hwnum}
\author{Kevin Browder \and Seth Bassetti \and Nathan Stouffer}
\collab{Kevin Browder, Seth Bassetti, Nathan Stouffer}
\date{due: 9 November 2020}

\begin{document}

\maketitle

This homework assignment should be submitted as a single PDF file to to Gradescope.

General homework expectations:
\begin{itemize}
    \item Homework should be typeset using LaTex.
    \item Answers should be in complete sentences and proofread.
    \item This HW can be submitted as an individual or as a group.
\end{itemize}

In any question that you are expected to provide an algorithm, you are expected to provide:
\begin{enumerate}
    \item Describe the problem in your own words, including describing what the input and output is.
    \item Describe, in paragraph form, the algorithm you propose.
    \item Provide a nicely formatted algorithm to solve the problem.
    \item Use a decrementing function to prove that algorithm terminates. OR  Give the runtime with justification.
    \item Prove partial correctness.
    In other words, if there is a loop or recursion, what is the loop/recursion invariant?
    Provide the proof.
    (Note: you only need to do this for the outer-most loop if there are nested loops).
\end{enumerate}

\nextprob
\collab{Kevin Browder, Seth Bassetti, Nathan Stouffer}

Chapter 7, Question 4, Part(a) (Maximum Weight Spanning Tree)

Describe and analyze an algorithm to compute the maximum-weight spanning tree of a given edge-weighted graph.

\paragraph{Answer}

% ============================================

Seth is working on this one

% ============================================


\nextprob
\collab{Kevin Browder, Seth Bassetti, Nathan Stouffer}

Chapter 8, Question 4, Part(a) (Removing an Edge)

For any edge $e$ in any graph $G$, let $G \setminus e$ denote the graph obtained by deleting $e$ from $G$.
Suppose we are given a graph $G$ and two vertices $s$ and $t$.
The replacement paths problem asks us to compute the shortest-path distance from $s$ to $t$ in $G \setminus e$, for every edge $e$ of $G$.
The output is an array of $E$ distances, one for each edge of $G$.

Suppose $G$ is a directed graph, and the shortest path from vertex $s$ to vertex $t$ passes through every vertex of $G$.
Describe an algorithm to solve this special case of the replacement paths problem in $O(E \log V)$ time.

\paragraph{Answer}

% ============================================

Seth is working on this

% ============================================

\nextprob
\collab{Kevin Browder, Seth Bassetti, Nathan Stouffer}

Chapter 10, Question 1, (Feasible Flow)

Let $f$ and $f'$ be two feasible $(s, t)$-flows in a flow network $G$, such that $|f'| > |f|$.
Prove that there is a feasible $(s, t)$-flow with value $|f'| - |f|$ in the residual network $G_f$.

\paragraph{Answer}

% ============================================

We must show that there is some feasible $(s,t)$-flow $g$ in the residual network $G_f$ that takes on the value $|g| = |f'| - |f|$.
We will prove this with a contradiction.
Suppose that there is no flow in $G_f$ that takes on the value $|f'| - |f|$. \parspace
First note that given a flow $h$ in $G_f$, we can construct a feasible flow $h': E \longrightarrow \R$ where $h'(u \to v) = f(u \to v) + h(u \to v)$ for $G$.
For $h'$ to be feasible, it must satisfy both the conservation and capacity constraints.
The function $h'$ is necessarily conservative because it is the sum of $f$ and $h$, both of which satisfy the conservation constraint for $G$ ($f$ by assumption and $h$ since it is a flow over the same vertices as $G$).
We claim that $h'$ also satisfies capacity.
Since $h'$ is certainly non-negative since $f$ and $h$ are non-negative, we must only show that $h'(u \to v) \leq c(u \to v)$.
To this end, note that
$$ h(u \to v) \leq c_f (u \to v) =
\begin{cases}
    c(u \to v) - f(u \to v)  & u \to v \in E \\
    f(v \to u)               & v \to u \in E \\
    0                        & otherwise
\end{cases} $$
Then we can say that $h'(u \to v) = f(u \to v) + h(u \to v) \leq f(u \to v) + c_f (u \to v)$.
Since $u \to v \in E$, we also have $f(u \to v) + c_f (u \to v) = f(u \to v) + c(u \to v) - f(u \to v) = c(u \to v)$.
Therefore, $h'$ also satisfies the capacity constraint and $h'$ is a flow on $G$. \parspace
We now return to the task at hand.
Since there is no flow in $G_f$ with value $|f'| - |f|$ there is also no flow in $G_f$ that takes on a value greater than $|f'| - |f|$ (if such a flow existed, we could taper the flow until it had value $|f'| - |f|$ so it cannot exist).
Then, for every flow $h$ in $G_f$, we must have $|h| < |f'| - |f|$.
Then the value of every flow for the original graph $G$ must be strictly less than $|f| + |h| = |f| + |f'| - |f| = |f'|$.
But this is a contradiction since we assumed there existed a flow $f'$ with value $|f'|$.
Since we found a contradiction when we assumed otherwise, it must be the case that there exists a flow $g$ in the residual network $G_f$ with value $|f'| - |f|$.

% ============================================


\nextprob
\collab{Kevin Browder, Seth Bassetti, Nathan Stouffer}

Chapter 10, Question 4, (Opposing Edges)

Let $G$ be a flow network that contains an opposing pair of edges $u \to v$ and $v \to u$, both with positive capacity.
Let $G'$ be the flow network obtained from $G$ by decreasing the capacities of both of these edge by $\min \{ c(u \to v), c(v \to u) \}$.

\begin{enumerate}[label=(\alph*)]
    \item Prove that every maximum $(s,t)$-flow in $G'$ is also a maximum $(s,t)$-flow in $G$.
    \item Prove that every minimum $(s,t)$-cut in $G$ is also a minimum $(s,t)$-cut in $G'$ and vice versa.
    \item Prove that there is at least one maximum $(s,t)$-flow in $G$ that is not a maximum $(s,t)$-flow in $G'$.
\end{enumerate}

\paragraph{Answer}

% ============================================

Stouff is working on this one.

% ============================================

\nextprob
\collab{Kevin Browder, Seth Bassetti, Nathan Stouffer}

Chapter 11, Question 6, (Mini-Golf)

The input consists of the $x,y$ coordinates of the $m$ corneres of the playing field, the $n$ starting points, and the $n$ holes.
Every hole can be made in a straight shot from it's starting point.
Assume that you can determine in constant time whether two line segements intersect, given the $x,y$ coordinates of their endpoints.
Describe and analyze an algorithm to compute a one-to-one correspondence between the starting points and the holes that meets the straight-line requirement, or to report that no such correspondence exists.

\paragraph{Answer}

% ============================================

\begin{enumerate}
    \item The goal of this problem is to assign each tee of a mini-golf course to an appropriate hole.
    Each tee can only finish at one hole and each hole must be the destination of exactly only one tee (the assignment of tees to holes is a bijection).
    We also require that each hole can be sunk with a straight shot from the starting point. \parspace
    The mini-golf course is represented as a sequnce of $(x,y)$ pairs stored in an array called $Border[1..b]$.
    We assume that $Border$ is arranged so that the endpoints of lines on the mini-golf course corrspond to the points at consecutive indices of $Border$.
    Additionally we assume that there is a line between the first and last entries of $Border$.
    The final assumption that we make about $Border$ is that connecting consecutive indices with a line forms a closed polygon. \parspace
    The starting points of the course are $(x,y)$ pairs stored in an array called $Tees[1..n]$ and the holes are $(x,y)$ pairs stored in an array called $Holes[1..n]$.
    We assume that $|tees| = |holes|$, otherwise it would be a strange golf course.
    Finally, we assume that every tee and hole has a distinct location inside the polygon formed by the border. \parspace
    The output will be an array called $matches[1..n]$ with $n$ entries.
    The $k^{th}$ element of the $matches$ is an integer representing the hole matched with the $k^{th}$ tee. \parspace
    We assume that we have the following functions $\textsc{Intersect}(p_1, q_1, p_2, q_2)$ and $\textsc{MaxFlow}(G, cap)$.
    $\textsc{Intersect}$ takes in two pairs of points and returns whether the line between $p_1$ and $p_1$ intersects with the line between $p_2$ and $q_2$.
    $\textsc{MaxFlow}$ takes in a graph $G$ and a capacity function $cap$ and uses the Ford-Fulkerson to compute the maximum flow for $G$.
    \item In words, we will solve this problem by forming a special directed graph, solving a max-flow problem on that graph and interpreting the results.
    First, we describe the graph.
    In this graph, the set of vertices is $\{ tee_1, tee_2, ... tee_n, hole_1, hole_2, ..., hole_n, s, t \}$ (one for every tee, one for every hole, and then a source and a target).
    Then, for each $tee_k$, we add an edge to every hole that can be reached with a straight line.
    Then we add an edge from the source vertex to every tee vertex as well as an edge from every hole vertex to the target vertex.
    Every edge is assigned capacity 1. \parspace
    We claim that solving the max-flow problem (using the Ford-Fulkerson algorithm) on this graph tells us a way to assign the tees to holes.
    We require the Ford-Fulkerson algorithm so that the flow is an integer-valued function.
    To verify our claim, note that each tee can only send out one unit of flow (since it receives exactly one unit of flow from the source).
    Additionally, each hole can only send out one unit of flow since its only outgoing edge is to the target with capacity 1.
    Then the capacity constraint of a flow forces the flow into each target to also be one.
    Since we are using Ford-Fulkerson, each tee will only be able to send its flow to a single target.
    If the flow returned from Ford-Fulkerson has capacity less than $n$, then there is no correspondence.
    If the flow returned from Ford-Fulkerson has capcacity $n$ (which is the only other case), then we match each tee with the hole that it sends its flow to.
    \item Here is the algorithm:
    \begin{algorithm}
        \textsc{GolfCourse}($Border[1..b]$, $Tees[1..n]$, $Holes[1..n]$) \\
        1. \hspace{1em} $G = (V,E) \gets \textsc{MakeGraph}(Border, Tees, Holes)$ \\
        2. \hspace{1em} $cap[1..n, 1..n]$   // initialize the capacity function (every entry is 0) \\
        3. \hspace{1em} for $u \to v \in E$ \\
        4. \hspace{2em}     $cap[u,v] \gets 1$ \\
        5. \hspace{1em} $maxflow \gets \textsc{MaxFlow}(G, cap)$ \\
        6. \hspace{1em} if ($|maxflow| < n$) \\
        7. \hspace{2em}     return ``No correspondence'' \\
        8. \hspace{1em} $matches[1..n]$ \\
        9. \hspace{1em} for $u \to v \in E$ \\
        10. \hspace{0.6em} if ($maxflow(u \to v) = 1$ and $u \neq s$ and $v \neq t$) \\
        11. \hspace{1.6em}     $matches[u] = v$ \\
        12. \hspace{0.6em} return $matches$ \\\\

        \textsc{MakeGraph}($Border[1..b]$, $Tees[1..n]$, $Holes[1..n]$) \\
        1. \hspace{1em} $V \gets \{ s, t \} \cup \{ tee_k \}_{k=1}^n \cup \{ hole_k \} _{k=1}^n$ \\
        2. \hspace{1em} $E \gets \{ s \to tee_k) \} _{k=1}^n \cup \{ (hole_k \to t) \} _{k=1}^n $ \\
        3. \hspace{1em} for $k \in \{ 1, 2, ..., n \} $ \\
        4. \hspace{2em}     for $j \in \{ 1, 2, ..., n \} $ \\
        5. \hspace{3em}         if (not \textsc{IntersectBorder}($Border$, $Tees[k]$, $Holes[j]$)) \\
        6. \hspace{4em}             $E \gets E \cup \{ tee_k \to hole_j \}$ \\
        7. \hspace{1em} return $(V, E)$ \\\\

        \textsc{IntersectBorder}($Border[1..b]$, $p_1$, $p_2$) \\
        1. \hspace{1em} for $i \in \{ 1, 2, ..., n-1 \}$ \\
        2. \hspace{2em}     if ($\textsc{Intersect}(Border[i], Border[i+1], p_1, p_2)$) \\
        3. \hspace{3em}         return True \\
        4. \hspace{1em} if ($\textsc{Intersect}(Border[1], Border[n], p_1, p_2)$) \\
        5. \hspace{2em}     return True \\
        6. \hspace{1em} return False
    \end{algorithm}
    \item Use a decrementing function to prove that algorithm terminates. OR  Give the runtime with justification.
    \item Prove partial correctness.
    In other words, if there is a loop or recursion, what is the loop/recursion invariant?
    Provide the proof.
    (Note: you only need to do this for the outer-most loop if there are nested loops).
\end{enumerate}

% ============================================

\nextprob
\collab{Kevin Browder, Seth Bassetti, Nathan Stouffer}

Find an algorithm discussed in a recent news article (over the past 12 months).
Choose ONE of the following:
\begin{enumerate}
    \item Look up the primary resource for this algorithm (likely to be a research paper).
    Compare/contrast the similarities and differences between the way the news article describes the problem and algorithm with the way that the primary resource describes it.
    \item If the algorithm itself is not given in the article, provide a prose description of the algorithm along with pseudocode.
    (This might require looking up the primary resource for the algorithm).
    \item Analyze the runtime of the algorithm.
    \item Prove the correctness of the algorithm.
\end{enumerate}

\paragraph{Answer}

% ============================================

TODO: your answer goes between these lines

% ============================================


\nextprob
\collab{Kevin Browder, Seth Bassetti, Nathan Stouffer}

Choose an algorithm that you analyzed on a homework in this class (can be this HW or a previous one).
Suppose you are a journalist writing about this break-through algorithm and write a one-page summary of the algorithm for a general audience.
Describing the problem that this algorithm solves and the applications of the problem should be highlighted (feel free to do some research).
Detail of the algorithm and proofs of correctness or runtime should be only given at a very high level.

\paragraph{Answer}

% ============================================

TODO: your answer goes between these lines

% ============================================





\end{document}
