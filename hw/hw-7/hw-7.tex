\documentclass{article}
\usepackage{../fasy-hw}
\usepackage{ wasysym }

%% UPDATE these variables:
\renewcommand{\hwnum}{7}
\title{Advanced Algorithms, Homework \hwnum}
\author{Kevin Browder \and Seth Bassetti \and Nathan Stouffer}
\collab{Kevin Browder, Seth Bassetti, Nathan Stouffer}
\date{due: 9 November 2020}

\begin{document}

\maketitle

This homework assignment should be submitted as a single PDF file to to Gradescope.

General homework expectations:
\begin{itemize}
    \item Homework should be typeset using LaTex.
    \item Answers should be in complete sentences and proofread.
    \item This HW can be submitted as an individual or as a group.
\end{itemize}

In any question that you are expected to provide an algorithm, you are expected to provide:
\begin{enumerate}
    \item Describe the problem in your own words, including describing what the input and output is.
    \item Describe, in paragraph form, the algorithm you propose.
    \item Provide a nicely formatted algorithm to solve the problem.
    \item Use a decrementing function to prove that algorithm terminates. OR  Give the runtime with justification.
    \item Prove partial correctness.
    In other words, if there is a loop or recursion, what is the loop/recursion invariant?
    Provide the proof.
    (Note: you only need to do this for the outer-most loop if there are nested loops).
\end{enumerate}

\nextprob
\collab{Kevin Browder, Seth Bassetti, Nathan Stouffer}

Chapter 7, Question 4, Part(a) (Maximum Weight Spanning Tree)

Describe and analyze an algorithm to compute the maximum-weight spanning tree of a given edge-weighted graph.

\paragraph{Answer}

% ============================================

TODO: your answer goes between these lines

% ============================================


\nextprob
\collab{Kevin Browder, Seth Bassetti, Nathan Stouffer}

Chapter 8, Question 4, Part(a) (Removing an Edge)

For any edge $e$ in any graph $G$, let $G \setminus e$ denote the graph obtained by deleting $e$ from $G$.
Suppose we are given a graph $G$ and two vertices $s$ and $t$.
The replacement paths problem asks us to compute the shortest-path distance from $s$ to $t$ in $G \setminus e$, for every edge $e$ of $G$.
The output is an array of $E$ distances, one for each edge of $G$.

Suppose $G$ is a directed graph, and the shortest path from vertex $s$ to vertex $t$ passes through every vertex of $G$.
Describe an algorithm to solve this special case of the replacement paths problem in $O(E \log V)$ time.

\paragraph{Answer}

% ============================================

TODO: your answer goes between these lines

% ============================================

\nextprob
\collab{Kevin Browder, Seth Bassetti, Nathan Stouffer}

Chapter 10, Question 1, (Feasible Flow)

Let $f$ and $f'$ be two feasible $(s, t)$-flows in a flow network $G$, such that $|f'| > |f|$.
Prove that there is a feasible $(s, t)$-flow with value $|f'| - |f|$ in the residual network $G_f$.

\paragraph{Answer}

% ============================================

TODO: your answer goes between these lines

% ============================================


\nextprob
\collab{Kevin Browder, Seth Bassetti, Nathan Stouffer}

Chapter 10, Question 4, (Opposing Edges)

Let $G$ be a flow network that contains an opposing pair of edges $u \to v$ and $v \to u$, both with positive capacity.
Let $G'$ be the flow network obtained from $G$ by decreasing the capacities of both of these edge by $\min \{ c(u \to v), c(v \to u) \}$.

\begin{enumerate}[label=(\alph*)]
    \item Prove that every maximum $(s,t)$-flow in $G'$ is also a maximum $(s,t)$-flow in $G$.
    \item Prove that every minimum $(s,t)$-cut in $G$ is also a minimum $(s,t)$-cut in $G'$ and vice versa.
    \item Prove that there is at least one maximum $(s,t)$-flow in $G$ that is not a maximum $(s,t)$-flow in $G'$.
\end{enumerate}

\paragraph{Answer}

% ============================================

Stouff is working on this one.

% ============================================

\nextprob
\collab{Kevin Browder, Seth Bassetti, Nathan Stouffer}

Chapter 11, Question 6, (Mini-Golf)

The input consists of the $x,y$ coordinates of the $m$ corneres of the playing field, the $n$ starting points, and the $n$ holes.
Every hole can be made in a straight shot from it's starting point.
Assume that you can determine in constant time whether two line segements intersect, given the $x,y$ coordinates of their endpoints.
Describe and analyze and algorithm to compute a one-to-one correspondence between the starting points and the holes that meets the straight-line requirement, or to report that no such correspondence exists.

\paragraph{Answer}

% ============================================

TODO: your answer goes between these lines

% ============================================

\nextprob
\collab{Kevin Browder, Seth Bassetti, Nathan Stouffer}

Find an algorithm discussed in a recent news article (over the past 12 months).
Choose ONE of the following:
\begin{enumerate}
    \item Look up the primary resource for this algorithm (likely to be a research paper).
    Compare/contrast the similarities and differences between the way the news article describes the problem and algorithm with the way that the primary resource describes it.
    \item If the algorithm itself is not given in the article, provide a prose description of the algorithm along with pseudocode.
    (This might require looking up the primary resource for the algorithm).
    \item Analyze the runtime of the algorithm.
    \item Prove the correctness of the algorithm.
\end{enumerate}

\paragraph{Answer}

% ============================================

TODO: your answer goes between these lines

% ============================================


\nextprob
\collab{Kevin Browder, Seth Bassetti, Nathan Stouffer}

Choose an algorithm that you analyzed on a homework in this class (can be this HW or a previous one).
Suppose you are a journalist writing about this break-through algorithm and write a one-page summary of the algorithm for a general audience.
Describing the problem that this algorithm solves and the applications of the problem should be highlighted (feel free to do some research).
Detail of the algorithm and proofs of correctness or runtime should be only given at a very high level.

\paragraph{Answer}

% ============================================

TODO: your answer goes between these lines

% ============================================





\end{document}
