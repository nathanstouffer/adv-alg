\documentclass{article}
\usepackage{../../hw/fasy-hw}
\usepackage{ wasysym }
\usepackage{amsmath}
\newcommand{\uvec}[1]{\boldsymbol{\hat{\textbf{#1}}}}

\renewcommand{\hwnum}{1}
\title{Advanced Algorithms Project, Part 1}
\author{Nathan Stouffer \and Kevin Browder \and Seth Basseti}
\collab{n/a}
\date{due: 22 October 2020}

\begin{document}

\maketitle

\newpage
\section{Problem-Statement}
According to Yifan Hu "the graph layout problem is one of finding a set of coordinates, $x = \{x_i | i \in V\}$,
with $x_i \in R2$ or $x_i \in R3$, for 2D or 3D layout, respectively, such that when the
graph $G$ is drawn with vertices placed at these coordinates, the drawing is visually
appealing." \cite{hu2005efficient} This problem is difficult because "visually appealing" is a subjective criteria. Different people have different interpretations of visually appealing. \\\\There is also no universal quantitative way to measure visually appealing which makes this a difficult problem for computer to solve but there are some metrics that can help define visually appealing a little bit better. One of these is minimal edge crossing. This is used because graphs get much harder to read and understand to the human eye when there are a whole bunch of edges crossing and going to different nodes. You need to look at the visualization a lot harder before you are able to get anything from the graph. Another method is taking the forces of nature. Humans are used to looking and processing visuals produced by the laws of nature so applying the physical laws to a graph is a natural step. This can be done by simulating forces between nodes based on their position and if they are connected with an edge. This can result in a very visually appealing graph and what we will exploring in this project with force directed graphs. 
Given two verticies and their locations, we compute the unit vector between them as $\uvec{u} = \dfrac{(x_j-x_i)}{\|x_j-x_i\|}$.
Note that $x_0$ denotes the set of vertex coordinates.
\begin{algorithm}
	\textsc{ForceDirectedAlgorithm}($G$, $x$, $tol$)\\
	1.\hspace{1em} converged = FALSE\\
	2.\hspace{1em} step = initial step length\\
	3.\hspace{1em} while (converged $=$ FALSE) \\
	4.\hspace{2em} $x_0 = x$\\
	5.\hspace{2em} for $i \in V$\\
	6.\hspace{3em} $f = 0$\\
	7.\hspace{3em} for each $(i \leftrightarrow j)$ \\
	8.\hspace{4em} $f = f + g_s(i, j)\uvec{u}$\\
	9.\hspace{3em} for $(j \neq i, j \in V)$\\
	10.\hspace{3.5em} $f = f + g_e(i, j)\uvec{u}$\\
	11.\hspace{2.5em} $x_i = x_i + step * (f / \| f \|)$\\
	12.\hspace{2.5em} step $=$ $0.9*$step\\
	13.\hspace{2.5em} if $(\|x - x_0\| < tol)$ \\
	14.\hspace{3.5em} converged = TRUE\\
	14.\hspace{1.5em} return x
\end{algorithm}


\section{Plus One}
For our plus one component of this project, we will create a visual presentation of the force-directed graph algorithm. 
To do this, we will implement the algorithm presented in Hu's paper. Utilizing this algorithm and additional Python packages for visualizations,
we will pick 4-5 planar graphs and display the algorithm through its iterations. We chose matplotlib as a Python plotting package
to display the graphs throughout iterations. We will begin by plotting each point with matplotlib and iterate through
the edges drawing lines with matplotlib to create a representation of the graph. Then, using a gif library, each graph can be saved as a frame
of the animation. After saving an individual frame to the animation, we will update the vertex positions using the "adaptive cooling" algorithm shown in Hu's paper. 
Each animation will begin with the vertices in a random layout and display the vertices changing positions throughout each iteration of this algorithm
as it optimizes the "energy" of the system. The animations will then be included in our presentation as a way of displaying,
visually, how this algorithm performs. By having a method of visualizing this algorithm's process, we can use it as an aid to
thoroughly explain the actual function and design of the algorithm.

\newpage
\bibliographystyle{acm}
\bibliography{P1}

\end{document}
