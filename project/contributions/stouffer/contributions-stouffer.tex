\documentclass{article}
\usepackage{../../../hw/fasy-hw}
\usepackage{ wasysym }
\usepackage{amsmath}
\newcommand{\uvec}[1]{\boldsymbol{\hat{\textbf{#1}}}}

\renewcommand{\hwnum}{1}
\title{Advanced Algorithms Project -- Contributions}
\author{Nathan Stouffer}
\collab{n/a}
\date{due: 16 November 2020}

\begin{document}

\maketitle

Our project was an iterative algorithm that computes a layout for displaying a graph.
The general structure of the algorithm was to apply forces to the vertices in an iterative manner and wait until convergence.
My contributions to our algorithms project was the following:
\begin{itemize}
    \item read through the paper
    \item do my best to understand the paper
    \item help with the implementation
    \item make some sweet gifs
    \item help with the video
\end{itemize}
I think the strengths of our project are the visualizations of the graphs converging.
We made a bunch of gifs with some different graphs and it is pretty cool to watch the graphs go from some random tangled state to a pretty pleasing output format.
The weakest part of our project is that the algorithm does not work on all graphs.
It seems to do really well on star-shaped graphs, fully connected graphs, and lattice graphs, but not so well on a circle graph and some other types.
\parspace
If we were to do this project again, I would want to try out other methods for determining the ``step size'' that our algorithm took.
The algorithm described by the paper kind of had contrived convergence.
Basically, it set an initial step size and forced every vertex to move at that step size and then it reduced the step size.
Then, of course the algorithm will eventually converge, because we are always decreasing the step size.
I think it would have been nice to explore other methods of updating the step size to see of the algorithm could have found a better layout.

\end{document}
