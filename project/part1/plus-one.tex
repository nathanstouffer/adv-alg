\section{Plus One}
For our plus one component of this project, we will create a visual presentation of the force-directed graph algorithm. 
To do this, we will implement the algorithm presented in Hu's paper. Utilizing this algorithm and additional Python packages for visualizations,
we will pick 4-5 planar graphs and display the algorithm through its iterations. We chose matplotlib as a Python plotting package
to display the graphs throughout iterations. We will begin by plotting each point with matplotlib and iterate through
the edges drawing lines with matplotlib to create a representation of the graph. Then, using a gif library, each graph can be saved as a frame
of the animation. After saving an individual frame to the animation, we will update the vertex positions using the ``adaptive cooling" algorithm shown in Hu's paper. 
Each animation will begin with the vertices in a random layout and display the vertices changing positions throughout each iteration of this algorithm
as it optimizes the ``energy" of the system. The animations will then be included in our presentation as a way of displaying,
visually, how this algorithm performs. By having a method of visualizing this algorithm's process, we can use it as an aid to
thoroughly explain the actual function and design of the algorithm.