\newpage
\section{Problem-Statement}
According to Yifan Hu ``the graph layout problem is one of finding a set of coordinates, $x = \{x_i | i \in V\}$,
with $x_i \in R^2$ or $x_i \in R^3$, for 2D or 3D layout, respectively, such that when the
graph $G$ is drawn with vertices placed at these coordinates, the drawing is visually
appealing." \cite{hu2005efficient} This problem is difficult because ``visually appealing" is a subjective criteria. Different people have different interpretations of visually appealing. \\\\There is also no universal quantitative way to measure visually appealing which makes this a difficult problem for computer to solve but there are some metrics that can help define visually appealing a little bit better. One of these is minimal edge crossing. This is used because graphs get much harder to read and understand to the human eye when there are a whole bunch of edges crossing and going to different nodes. You need to look at the visualization a lot harder before you are able to get anything from the graph. Another method is taking the forces of nature. Humans are used to looking and processing visuals produced by the laws of nature so applying the physical laws to a graph is a natural step. This can be done by simulating forces between nodes based on their position and if they are connected with an edge. This can result in a very visually appealing graph and what we will exploring in this project with force directed graphs. 